Este tópico abordará os testes realizados nos modelos de aprendizado de máquina para identificação e classificação de oócitos bovinos.
\section{Avaliação do Modelo de Identificação}


A \autoref{tab:metricas_ap} demonstra as informações geradas pelo método de avaliação \textit{COCOEValuator}, o avaliador nativo do \textit{Detectron2}.

% Definições de cores (certifique-se de que estão no PREÂMBULO ou antes da tabela)
\definecolor{mediumgray}{gray}{0.75}  % Cinza mediano
\definecolor{lightgray}{gray}{0.9}    % Cinza claro

\begin{table}[!htbp]
	\footnotesize
	\centering
	\caption{Métricas de Precisão Média (\textit{Average Precision} - AP) do COCOEvaluator.}
	
	% As colunas l X X X definem a estrutura da tabela
	\begin{tabularx}{\textwidth}{l X X X}
		\hline
		\rowcolor{mediumgray}
		\textbf{Métrica} & \textbf{Descrição} & \textbf{Caixas Delimitadoras} & \textbf{Segmentação} \\ 
		\hline
		
		\rowcolor{lightgray}
		$\text{AP}$ & AP Principal ($\text{IoU}=0.50$ a $0.95$, média) & $\mathbf{80.974\%}$ & $\mathbf{78.405\%}$ \\
		
		\rowcolor{white}
		$\text{AP}^{50}$ & AP com $\text{IoU} \ge 0.50$ (Detecção "fácil") & $97.783\%$ & $97.783\%$ \\
		
		\rowcolor{lightgray}
		$\text{AP}^{75}$ & AP com $\text{IoU} \ge 0.75$ (Detecção "rigorosa") & $96.033\%$ & $94.945\%$ \\
		
		
		\rowcolor{white}
		$\text{AP}_M$ & AP para objetos médios & $81.938\%$ & $79.493\%$ \\
		
		\rowcolor{lightgray}
		$\text{AP}_L$ & AP para objetos grandes & $81.022\%$ & $78.314\%$ \\
		\hline
	\end{tabularx}
	\caption*{Fonte: Elaboração Própria.}
	\label{tab:metricas_ap}
\end{table}


Primeiramente, é preciso explicar os termos da tabela para compreênde-la adequadamente. Dessa forma, \textit{Insertection over Union} (Iou), que é a base para todas as medidas, serve para calcular a proporção entre a área de intersecção e a área da união entre o espaço previsto pelo modelo e a anotação real feita no pré-processamento. Portanto, o resultado ideal para essa proporção é 1.0 e quanto mais perto desse valor, melhor. Isso acontece porque um resultado de 1.0 indica que a área prevista está exatamente onde a anotação foi feita. Essa fórmula pode ser visualizada na Figura \ref{fig:formula}.


\begin{figure}[H]
	\centering
	\caption{Fórmula d IoU}
	\includegraphics[width=0.8\textwidth]{images/ia/formula.png}
	\caption*{\textbf{Fonte: \citeonline{rosebrock2016iou}.}} 
	\label{fig:formula}
\end{figure}

Além do mais, as caixas delimitadoras definem o quão bem o modelo consegue desenhar um retângulo ao redor do oócito, enquanto a segmentação verifica o quão bem o modelo desenha uma máscara nos contornos exatos do oócito, ou seja, \textit{pixel} por \textit{pixel}. E, por fim, \textit{Average Precision} (AP), é simplesmente o resultado médio realizado a partir de vários valores diferentes de IoU. 

Dito isso, pode-se destrinchar os elementos da tabela:

\begin{itemize}
	\item \textbf{AP (IoU=0.50 a 0.95, média)}: Demonstra uma alta precisão média do modelo (>75\%). A Detecção (80.974\%) é ligeiramente maior que a Segmentação (78.405\%). Isso significa que o modelo possui um pouco mais de facilidade em delimitar os retângulos do que os contornos dos oócitos;
	
	\item \textbf{{$\text{AP}^{50}$} (IoU$\geq$0.50)}: Há uma excelente Localização porque o modelo consegue identificar praticamente todos os objetos corretamente nos dois processos com um IoU moderado;
	
	\item \textbf{{$\text{AP}^{75}$} (IoU$\geq$0.75)}: Mesmo com um limite consideravelmente mais rigoroso, a delimitação as caixas e dos contornos continua excelente e precisa;
	
	\item \textbf{{$\text{AP}_{M}$} (Objetos Médios)}: Há uma precisão adequada para ambos os processos quando se trata de oócitos de tamanha médio, no entanto, a delimitação da caixa é ligeiramente melhor;
	
	\item \textbf{{$\text{AP}_{L}$} (Objetos Grandes)}: Apesar da precisão continuar boa, há uma pequena queda em comparação à oócitos médios;
\end{itemize}

Dessa forma, conclui-se que o modelo de identificação de oócitos bovinos está funcionando de maneira ideal.  

\section{Avaliação do Modelo de Classificação}

A \autoref{tab:metricas322} demonstra as informações geradas pelo método de avaliação \textit{COCOEValuator}.


\definecolor{mediumgray}{gray}{0.75} % Cinza mediano
\definecolor{lightgray}{gray}{0.9}   % Cinza claro



\begin{table}[!htbp]
	\footnotesize
	\centering
	\caption{Métricas Globais de Precisão Média (\textit{Average Precision} - AP) do Modelo de Classificação.}
	
	% 4 colunas: l X X X
	\begin{tabularx}{\textwidth}{l X X X}
		\hline
		\rowcolor{mediumgray}
		\textbf{Métrica} & \textbf{Descrição} & \textbf{Detecção (bbox AP)} & \textbf{Segmentação (segm AP)} \\
		\hline
		
		\rowcolor{white}
		$\text{AP}$ & AP Principal ($\text{IoU}=0.50$ a $0.95$) & $\mathbf{88.128\%}$ & $\mathbf{94.243\%}$ \\
		
		\rowcolor{lightgray}
		$\text{AP}^{50}$ & AP com $\text{IoU} \ge 0.50$ (Detecção "Fácil") & $100.000\%$ & $100.000\%$ \\
		
		\rowcolor{white}
		$\text{AP}^{75}$ & AP com $\text{IoU} \ge 0.75$ (Detecção "Rigorosa") & $100.000\%$ & $100.000\%$ \\
		
		\rowcolor{lightgray}
		$\text{AP}_M$ & AP para objetos médios ($32^2 < \text{Area} < 96^2$) & $88.762\%$ & $92.673\%$ \\
		
		\rowcolor{white}
		$\text{AP}_L$ & AP para objetos grandes ($\text{Area} > 96^2$) & $88.960\%$ & $95.384\%$ \\
		\hline
	\end{tabularx}
	\caption*{Fonte: Elaboração Própria (Baseado nos resultados do COCOEvaluator).}
	\label{tab:metricas322}
\end{table}

Dessa forma, é possível destrinchar os elementos dessa tabela:

\begin{itemize}
	\item \textbf{AP (IoU=0.50 a 0.95, média)}: O modelo atinge uma alta precisão no geral, mas possui mais facilidade com a segmentação;
	
	\item \textbf{{$\text{AP}^{50}$} (IoU$\geq$0.50)}: Há uma localização perfeita, o que demonstra que houve uma memorização do conjunto de validação;
	
	\item \textbf{{$\text{AP}^{75}$} (IoU$\geq$0.75)}: Apesar do limiar rigoroso, a localização perfeita permanece, algo que corrobora com a memorização do conjunto de validação;
	
	\item \textbf{{$\text{AP}_{M}$} (Objetos Médios)}: Grande perfomance, especialmente na segmentação;
	
	\item \textbf{{$\text{AP}_{L}$} (Objetos Grandes)}: O modelo lida melhor com objetos de interesse grandes;
\end{itemize}

A principal conclusão acerca desses dados é que o modelo possui um problema de \textit{overfitting}, ou seja, o modelo não conseguiu generalizar seu aprendizado. Ele apenas memorizou as imagens que foram usadas para validá-lo. Por conta disso, uma série de estratégias foram utilizadas para lidar com essa óbice. Elas são:

\begin{itemize}
	\item \textbf{Aumento do \textit{Dataset}}: A primeira versão do modelo foi treinada com, em média, 16 oócitos para classificação. Esse valor foi escolhido porque a classificação 3 de oócitos, que foi a mais difícil de achar, continha apenas 16 oócitos em todo o \textit{dataset}, enquanto as outras classificações possuíam consideravelmente mais imagens. Como o \textit{overfitting} aconteceu, foi decidido alterar o balanceamento e utilizar o total de imagens que cada classe possuía para si;
	
	\item \textbf{Regularização L2}: A variável \textit{Weight Decay} presente na configuração do treinamento foi diminuída com o intuito de aumentar a generalização do aprendizado, tendo em vista que um grande valor torna o treinamento muito sensível para variações entre diferentes imagens de oócitos;
	
	\item \textbf{Regularização de Lote}: Diminuir a variável \textit{BATCH\_SIZE\_PER\_IMAGE} força o treinamento a se adaptar para características mais amplas e combater a memorização das imagens de validação;
	
	\item \textbf{Otimização de Treinamento}: Reduzir a variável \textit{BASE\_LR} serve pra estabilizar o aprendizado e diminuir a intensidade que o modelo se corrige durante o processo de treinamento;
	
	\item \textbf{Parada/Controle}: Essa estratégia permite manipular a duração do treinamento. Ele verifica a partir de qual ponto o \textit{overfitting} começa através do monitoramento da variável AP e interrompe o treinamento assim que o problema de memorização começa.
\end{itemize}

Portanto, nota-se como nenhuma das estratégias foi suficiente para resolver o \textit{overfitting}. Conclui-se que, para resolver esse problema, é necessário que haja um \textit{dataset} consideravelmente maior e que contenha imagens de alta qualidade dos oócitos isolados. A empresa parceira, BioInova, não conseguiu proporcionar essas imagens em tempo hábil e não havia como realizar essa coleta por conta própria tendo em vista que não há estrutura para isso na região.
