Este tópico abordará os resultados obtidos e a discussão sobre eles acerca dos modelos de aprendizado de máquina para identificação e classificação de oócitos bovinos.
\section{Resultados do Modelo de Identificação}


A \autoref{tab:metricas_ap} demonstra os resultados obtidos pelo detectron2 através do COCO Evaluator, métrica padrão dessa rede.

% Definições de cores (certifique-se de que estão no PREÂMBULO ou antes da tabela)
\definecolor{mediumgray}{gray}{0.75}  % Cinza mediano
\definecolor{lightgray}{gray}{0.9}    % Cinza claro

\begin{table}[!htbp]
	\footnotesize
	\centering
	\caption{Métricas de Precisão Média (\textit{Average Precision} - AP) do COCOEvaluator.}
	
	% As colunas l X X X definem a estrutura da tabela
	\begin{tabularx}{\textwidth}{l X X X}
		\hline
		\rowcolor{mediumgray}
		\textbf{Métrica} & \textbf{Descrição} & \textbf{Caixas Delimitadoras} & \textbf{Segmentação} \\ 
		\hline
		
		\rowcolor{lightgray}
		$\text{AP}$ & AP Principal ($\text{IoU}=0.50$ a $0.95$, média) & $\mathbf{80.974\%}$ & $\mathbf{78.405\%}$ \\
		
		\rowcolor{white}
		$\text{AP}^{50}$ & AP com $\text{IoU} \ge 0.50$ (Detecção "fácil") & $97.783\%$ & $97.783\%$ \\
		
		\rowcolor{lightgray}
		$\text{AP}^{75}$ & AP com $\text{IoU} \ge 0.75$ (Detecção "rigorosa") & $96.033\%$ & $94.945\%$ \\
		
		
		\rowcolor{white}
		$\text{AP}_M$ & AP para objetos médios & $81.938\%$ & $79.493\%$ \\
		
		\rowcolor{lightgray}
		$\text{AP}_L$ & AP para objetos grandes & $81.022\%$ & $78.314\%$ \\
		\hline
	\end{tabularx}
	\caption*{Fonte: Elaboração Própria.}
	\label{tab:metricas_ap}
\end{table}

Dessa forma, pode-se destrinchar os elementos da tabela:

\begin{itemize}
	\item \textbf{AP (IoU=0.50 a 0.95, média)}: Demonstra uma alta precisão média do modelo (>75\%). A Detecção (80.974\%) é ligeiramente maior que a Segmentação (78.405\%);
	
	\item \textbf{{$\text{AP}^{50}$} (IoU$\geq$0.50)}: Há uma excelente localização porque o modelo consegue identificar praticamente todos os objetos corretamente nos dois processos com um IoU moderado;
	
	\item \textbf{{$\text{AP}^{75}$} (IoU$\geq$0.75)}: Mesmo com um limite consideravelmente mais rigoroso, a delimitação as caixas e dos contornos continua excelente e precisa;
	
	\item \textbf{{$\text{AP}_{M}$} (Objetos Médios)}: Há uma precisão adequada para ambos os processos quando se trata de oócitos de tamanha médio;
	
	\item \textbf{{$\text{AP}_{L}$} (Objetos Grandes)}: Apesar da precisão continuar boa, há uma pequena queda em comparação à oócitos médios;
\end{itemize}

O modelo de identificação apresentou uma eficiência excelente em seu desempenho por conta de sua alta precisão média mesmo nas métricas mais rigorosas. Além do mais, ele também é altamente idôneo tanto para as caixas delimitadoras quanto para as segmentações dos oócitos.

No entanto, uma comparação entre os valores revela uma distinção: a delimitação da caixa é ligeiramente mais precisa do que a tarefa de delimitação do contorno, apesar das duas possuírem bons resultados. Ademais, há um ótimo desempenho para identificação de objetos médios, enquanto há uma pequena queda para os objetos grandes. 

Por fim, esse modelo é um sucesso técnico tendo em vista que proporciona resultados com alta confiabilidade para ambos os principais processos de detecção e segmentação. Sua robustez garante uma alta aplicabilidade em ambientes práticos. 

\section{Resultados do Modelo de Classificação}

A \autoref{tab:metricas322} demonstra os resultados obtidos pelo método de avaliação \textit{COCOEValuator}.


\definecolor{mediumgray}{gray}{0.75} % Cinza mediano
\definecolor{lightgray}{gray}{0.9}   % Cinza claro



\begin{table}[!htbp]
	\footnotesize
	\centering
	\caption{Métricas Globais de Precisão Média (\textit{Average Precision} - AP) do Modelo de Classificação.}
	
	% 4 colunas: l X X X
	\begin{tabularx}{\textwidth}{l X X X}
		\hline
		\rowcolor{mediumgray}
		\textbf{Métrica} & \textbf{Descrição} & \textbf{Detecção (bbox AP)} & \textbf{Segmentação (segm AP)} \\
		\hline
		
		\rowcolor{white}
		$\text{AP}$ & AP Principal ($\text{IoU}=0.50$ a $0.95$) & $\mathbf{88.128\%}$ & $\mathbf{94.243\%}$ \\
		
		\rowcolor{lightgray}
		$\text{AP}^{50}$ & AP com $\text{IoU} \ge 0.50$ (Detecção "Fácil") & $100.000\%$ & $100.000\%$ \\
		
		\rowcolor{white}
		$\text{AP}^{75}$ & AP com $\text{IoU} \ge 0.75$ (Detecção "Rigorosa") & $100.000\%$ & $100.000\%$ \\
		
		\rowcolor{lightgray}
		$\text{AP}_M$ & AP para objetos médios ($32^2 < \text{Area} < 96^2$) & $88.762\%$ & $92.673\%$ \\
		
		\rowcolor{white}
		$\text{AP}_L$ & AP para objetos grandes ($\text{Area} > 96^2$) & $88.960\%$ & $95.384\%$ \\
		\hline
	\end{tabularx}
	\caption*{Fonte: Elaboração Própria (Baseado nos resultados do COCOEvaluator).}
	\label{tab:metricas322}
\end{table}

Dessa forma, é possível destrinchar os elementos dessa tabela:

\begin{itemize}
	\item \textbf{AP (IoU=0.50 a 0.95, média)}: O modelo atinge uma alta precisão no geral, mas possui mais facilidade com a segmentação;
	
	\item \textbf{{$\text{AP}^{50}$} (IoU$\geq$0.50)}: Há uma localização perfeita, o que demonstra que houve uma memorização do conjunto de validação;
	
	\item \textbf{{$\text{AP}^{75}$} (IoU$\geq$0.75)}: Apesar do limiar rigoroso, a localização perfeita permanece, algo que corrobora com a memorização do conjunto de validação;
	
	\item \textbf{{$\text{AP}_{M}$} (Objetos Médios)}: Grande perfomance, especialmente na segmentação;
	
	\item \textbf{{$\text{AP}_{L}$} (Objetos Grandes)}: O modelo lida melhor com objetos de interesse grandes;
\end{itemize}

O modelo de classificação apresentou resultados perfeitos, ou seja, alcançou 100\% em {$\text{AP}^{50}$} e {$\text{AP}^{75}$} no conjunto de validação. No entanto, houve uma queda considerável para 88.128\% (Detecção) e 94.243\% (Segmentação) na métrica de AP principal e mais exigente. De acordo com \citeonline{ying2019overview}, esse contraste entre as métricas é um claro indicativo de que o modelo memorizou os objetos de interesse do conjuto de validação ao invés de generalizar seu aprendizado. 

Além do mais, as métricas com porcentagens perfeitas indica que o principal problema foi o aprendizado de ruído \cite{ying2019overview}. Esse problema ocorre quando o \textit{dataset} de treinamento não possui um tamanho adequado para as necessidades do modelo. Essa escassez aumenta consideravelmente as chances de acontecer uma memorização. Embora uma estratégia de aumento de dados tenha sido aplicada, ela não foi suficiente. 





