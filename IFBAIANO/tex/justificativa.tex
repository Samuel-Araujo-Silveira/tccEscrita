
A economia brasileira é fortemente marcada por atividades econômicas que envolvem bovinos. O Brasil detém o segundo maior rebanho do mundo, é o segundo maior produtor de carne bovina \cite{FAOSTAT2022c}, é o maior exportador e é o quinto maior produtor de leite \cite{FAO2019}. A criação de bovinos é muito importante para a geração de empregos, para a economia nacional e para a produção de alimentos altamente nutritivos para o Brasil e para o mundo \cite{teixeira2014trajetoria}.

Isso se deve ao fato de que o Brasil, por estar localizado em clima tropical, não pôde copiar as tecnologias de pecuária de países de clima temperado, como os Estados Unidos da América. Essa barreira impulsionou o avanço tecnológico e fez com que os brasileiros desenvolvessem suas próprias tecnologias, que acabaram servindo de modelo para outros países de clima tropical \cite{Camargo2022}.

As raças bovinas mais produtivas evoluíram e foram melhoradas geneticamente em países de clima temperado, inicialmente na Europa, posteriormente na América do Norte e em outras regiões de clima temperado. Como a maior parte do Brasil está localizada na faixa tropical, é necessário o desenvolvimento de animais adaptados à esta condição ambiental \cite{oosting2014development}, o que está sendo feito com enorme pioneirismo e competência por parte dos setores de produção de carne e de leite do país. Para acelerar este melhoramento genético, as técnicas de reprodução assistidas são imprescindíveis, como a inseminação artificial e a produção \textit{in vitro} de embriões \cite{sales2024evolution}.

A produção \textit{in vitro} possui várias etapas no seu funcionamento e uma delas é a classificação de células reprodutoras, como os gametas femininos nomeados de oócitos \cite{MELLO}. Atualmente, essa etapa é realizada de forma manual \cite{Shivaani2024}, algo que ocasiona em um grande número de horas de trabalho por parte dos profissionais da medicina veterinária que realizam esse serviço. 

Nesse contexto, surgem sistemas de informação que utilizam modelos inteligentes que pode aprimorar muito a eficiência de procedimentos já existentes e realizados por humanos. Dessa forma, percebe-se como há espaço para uma parceria entre pecuária e computação com o objetivo de agilizar e aprimorar a produção \textit{in vitro}.

Portanto, essa proposta se justifica pelo impacto social e econômico que ela pode produzir para a pecuária nacional.