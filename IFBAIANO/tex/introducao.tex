A pecuária possui grande relevância em diversos âmbitos nacionais, seja na nutrição do povo brasileiro, na geração de empregos ou em relações comerciais internacionais \cite{EmbrapaCarneBovina2024}. Para sustentar essa importância, o Brasil possui 1,8 milhão de propriedades leiteiras, 3,6 milhões de empregos gerados apenas pelo segmento da pecuária leiteira \cite{NettoGomes2021} e a exportação de carne bovina representa 6\% do \sigla{PIB}{Produto Interno Bruto} nacional \cite{EmbrapaCarneBovina2024}.

De acordo com \citeonline{Oliveira2014}, a base da pecuária e suas ramificações, não só do Brasil, mas do mundo, é formada por técnicas de reprodução animal assistida. Isso acontece porque essas técnicas são úteis tanto para aumentar consideravelmente a quantidade de cabeças de gado, quanto para aprimorar a qualidade genética dos animais. 

Nesse contexto, existe a \sigla{PIV}{Produção \textit{in vitro}} como uma importante técnica de reprodução assistida. Esse método permite que haja uma seleção das melhores células reprodutoras antes de serem enviadas para os laboratórios responsáveis por finalizar o procedimento. Essa seleção precisa ser precedida de uma classificação morfológica e funcional, que ocorre logo após a coleta das células de interesse \cite{MELLO}. Para realizar a classificação, é necessário um profissional da medicina veterinária para analisar as características de célula por célula. Portanto, nota-se como esse processo está dependente de um trabalho longo, manual e condicionado à profissionais capacitados.

Outra área de grande relevância para a atualidade é a \sigla{TI}{Tecnologia da Informação}. Um de seus princípios fundamentais consiste em atuar de forma integrada com outras áreas do conhecimento, visando automatizar processos ou aprimorar a eficiência de procedimentos tradicionalmente manuais. Tal contribuição é possível porque os sistemas computacionais são capazes de processar grandes volumes de dados em curto tempo e de forma isenta do enviesamento característico das análises humanas \cite{henrique2024processamento}

Através de tecnologias modernas de inteligência artificial, é possível desenvolver modelos de aprendizado de máquina que classifiquem automaticamente oócitos, as células reprodutoras femininas, coletados em anima. Com uma classificação aprimorada, é possível também melhorar a etapa de seleção dos oócitos que serão os enviados para os laboratórios de PIV. Dessa forma, o aprimoramento genético oriundo desse processo pode ocasionar uma grande melhoria na qualidade do gado brasileiro \cite{Mueller2022}.

Por fim, a automatização do reconhecimento e classificação de oócitos bovinos é realizada através da coleta de imagens dessas células reprodutoras em laboratório, pré-processamento de cada uma delas e treinamento de dois modelos diferentes: um para identificação e outro para classificação.

