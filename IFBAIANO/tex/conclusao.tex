A construção de um modelo adequado de aprendizado de máquina para identificação de oócitos bovinos e a análise dos problemas enfrentados para a construção do modelo de classificação são as principais contribuições desse trabalho. Isso representa um grande avanço para automatizar a etapa de classificação da PIV, algo de grande importância para a economia brasileira. 


O desenvolvimento desses modelos foi dependente de tecnologias específicas, como visão computacional, processamento de imagens, aprendizado de máquina e programação. Elas permitiram pré-processar as imagens, rotulá-las, processá-las e, por fim, a união das funcionalidades através de um algoritmo. Os resultados obtidos representam uma contribuição significativa para futuros trabalhos que queiram unir a tecnologia da informação com a pecuária.

Para a contribuição na construção de modelos futuros, é importante considerar a importância de uma alta qualidade das imagens, algo que permite uma visão adequada dos fatores que compõem as métricas da qualidade de um oócito. 

Enfim, percebe-se como a união de diferentes áreas da computação, como visão computacional, processamento de imagens e aprendizado de máquina, possuem um grande potencial para desenvolver trabalhos em parceria com outras áreas do conhecimento humano. Demonstrou-se com sucesso o potencial e capacidade que a tecnologia da informação possui para automatizar e, como consequência, aumentar a eficiência de procedimentos que ainda precisam ser feitas de maneira manual.

