A construção de um modelo adequado de aprendizado de máquina para identificação de oócitos bovinos e a análise dos problemas enfrentados para a construção do modelo de classificação são as principais contribuições desse trabalho. Isso representa um grande avanço para automatizar a etapa de classificação da PIV, algo de grande importância para a economia brasileira. 

Para um bom desenvolvimento deste trabalho, técnicas de aprendizado de máquina e visão computacional foram estudadas e aplicadas com rigor técnico para que o máximo pudesse ser extraído dos recursos disponíveis para o trabalho. Além do mais, houve a necessidade de obter uma compreensão de técnicas de processamento digital de imagens, que compõem um ramo da visão computacional, para que os treinamentos pudessem ocorrer adequadamente. 

A partir desses estudos e aplicações, conclui-se que há um grande potencial para a capacidade de generalização de métodos aplicada a reprodução animal assistida tendo em vista os ótimos resultados obtidos para o modelo de identificação. Adicionalmente, redes neurais convolucionais mostraram-se de extrema importância para o bom aprendizado dos treinamentos. Mais especificamente, a Mask R-CNN demonstrou-se com uma técnica bastante pertinente para o trabalho por conta da sua característica de segmentação de instância que permite delimitar os contornos dos oócitos ao nível de \textit{pixel}.

Dito isso, a principal conclusão acerca dos resultados o modelo de classificação é que ele possui um problema de \textit{overfitting}, ou seja, o modelo não conseguiu generalizar seu aprendizado. Ele apenas memorizou as imagens que foram usadas para validá-lo.

Portanto, nota-se como nenhuma das estratégias foi suficiente para resolver o \textit{overfitting}. Conclui-se que, para resolver esse problema, é necessário que haja um \textit{dataset} consideravelmente maior e que contenha imagens de alta qualidade dos oócitos isolados. A empresa parceira, BioInova, não conseguiu proporcionar essas imagens em tempo hábil e não havia como realizar essa coleta por conta própria tendo em vista que não há estrutura para isso na região.

