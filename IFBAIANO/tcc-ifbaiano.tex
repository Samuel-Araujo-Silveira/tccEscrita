% ------------------------------------------------------------------------
% ------------------------------------------------------------------------
% ICMC: Modelo de Trabalho Acadêmico (tese de doutorado, dissertação de
% mestrado e trabalhos monográficos em geral) em conformidade com 
% ABNT NBR 14724:2011: Informação e documentação - Trabalhos acadêmicos -
% Apresentação
% ------------------------------------------------------------------------
% ------------------------------------------------------------------------

% Opções: 
%   Qualificação          = qualificacao 
%   Curso                 = doutorado/mestrado
%   Situação do trabalho  = pre-defesa/pos-defesa (exceto para qualificação)
%   Versão para impressão = impressao
\documentclass[qualificacao]{packages/icmc}



% ---------------------------------------------------------------------------
% Pacotes Opcionais
% ---------------------------------------------------------------------------
\usepackage{rotating}           % Usado para rotacionar o texto
\usepackage[all,knot,arc,import,poly]{xy}   % Pacote para desenhos gráficos
% Este pacote pode conflitar com outros pacotes gráficos como o ``pictex''
% Então é necessário usar apenas um dos pacotes conflitantes

\newcommand{\VerbL}{0.52\textwidth}
\newcommand{\LatL}{0.42\textwidth}

% ---------------------------------------------------------------------------






% ---
% Informações de dados para CAPA e FOLHA DE ROSTO
% ---
% Tanto na capa quanto nas folhas de rosto apenas a primeira letra da primeira palavra (ou nomes próprios) devem estar em letra maiúscula, todas as demais devem ser em letra minúscula.
\tituloPT{Classificação de Oócitos bovinos através de Aprendizado de Máquina}
\tituloEN{Classification of Bovine Oocytes through Machine Learning}
\autor[Silveira, S. A.]{Samuel Araújo Silveira}
\genero{M} % Gênero do autor (M = Masculino / F = Feminino)
\orientador[Orientador]{Prof. Dr.}{Carlos Anderson Oliveira Silva}
\coorientador{Prof. Dr.}{José Assunção Silveira Júnior}
\curso{CCMC}

% ===== MODIFICAÇÃO AQUI =====


\data{03}{11}{2024} % Data do depósito
\idioma{PT} % Idioma principal do documento (PT = português / EN = inglês)
% ---



% ---
% RESUMOS
% ---

% Resumo em PORTUGUÊS
% conter no máximo 500 palavras
% conter no mínimo 1 e no máximo 5 palavras-chave
\textoresumo[brazil]{
	A pecuária ocupa uma posição relevante no contexto socioeconômico brasileiro, já que o Brasil está entre os maiores exportadores de leite e carne bovina do mundo. O que torna isso possível são os métodos de reprodução animal assistida, que garantem um maior número de descendentes com boas genéticas. Um desses métodos é a produção \textit{in vitro} cujas duas etapas iniciais são as de coleta e classificação de células reprodutoras. Atualmente, a etapa de classificação precisa ser feita de forma manual e apenas por profissionais capacitados, o que causa um prejuízo grave na eficiência do processo. Dessa forma, este trabalho desenvolveu
	um modelo de aprendizado de máquina capaz de classificar oócitos bovinos com o objetivo de aprimorar a eficiência de
	procedimentos de reprodução animal assistida. Os métodos empregados para automatizar esse processo são baseados em redes neurais convolucionais e processamento digital de imagens. Com isso, pretende-se
	reconhecer e classificar oócitos bovinos, aumentar
	a eficiência da produção \textit{in vitro} e aprimorar a qualidade genética do gado brasileiro
	através de uma melhor classificação e seleção dos gametas femininos bovinos. A pesquisa
	demonstra uma colaboração entre computação e pecuária que pode superar práticas tradicionais
	através da proposta de soluções mais eficientes.
}{Aprendizado de Máquina, Classificação de Oócitos, Reprodução Animal Assistida, Redes Neurais Convolucionais, Agropecuária}

% resumo em INGLÊS
% conter no máximo 500 palavras
% conter no mínimo 1 e no máximo 5 palavras-chave
\textoresumo[english]{
	Livestock farming occupies a relevant position in the Brazilian socioeconomic context, since Brazil is among the largest exporters of milk and beef in the world. What makes this possible are the methods of assisted animal reproduction, which guarantee a greater number of descendants with good genetics. One of these methods is \textit{in vitro} production, whose two initial stages are the collection and classification of reproductive cells. Currently, the classification stage must be done manually and only by trained professionals which causes a massive problem in the efficiency of this process. Thus, this work developed a system capable of classifying bovine oocytes with the objective of improving the efficiency of assisted animal reproduction procedures. The methods used to automate this process are based on convolutional neural networks and digital image processing. With this, the aim is to recognize and classify bovine oocytes n, increase the efficiency of \textit{in vitro} production, and improve the genetic quality of Brazilian cattle through better classification and selection of female bovine gametes. The research
	demonstrates a collaboration between computing and livestock that can overcome traditional practices
	by proposing more efficient solutions.
}{Machine Learning, Oocyte Classification, Assisted Animal Reproduction, Convolutional Neural Networks, animal husbandry}


% ----------------------------------------------------------
% ELEMENTOS PRÉ-TEXTUAIS
% ----------------------------------------------------------


% Inserir a ficha catalográfica
\incluifichacatalografica{tex/pre-textual/ficha-catalografica/ficha-catalografica}

% DEDICATÓRIA / AGRADECIMENTO / EPÍGRAFE
%\textodedicatoria*{tex/pre-textual/dedicatoria/dedicatoria}
%\textoagradecimentos*{tex/pre-textual/agradecimentos/agradecimentos}
\textoepigrafe*{tex/pre-textual/epigrafe/epigrafe}

% Inclui a lista de figuras
\incluilistadefiguras

% Inclui a lista de tabelas
\incluilistadetabelas

% Inclui a lista de quadros
%\incluilistadequadros

% Inclui a lista de algoritmos
%\incluilistadealgoritmos

% Inclui a lista de códigos
%\incluilistadecodigos

% Inclui a lista de siglas e abreviaturas
\incluilistadesiglas

% Inclui a lista de símbolos
%\incluilistadesimbolos

% ----
% Início do documento
% ----
\begin{document}
	

	% ----------------------------------------------------------
	% ELEMENTOS TEXTUAIS
	% ----------------------------------------------------------
	\textual
	
	\chapter{Introdução}
	\label{chapter:introducao}
	A pecuária possui grande relevância em diversos âmbitos nacionais, seja na nutrição do povo brasileiro, na geração de empregos ou em relações comerciais internacionais \cite{EmbrapaCarneBovina2024}. Para sustentar essa importância, o Brasil possui 1,8 milhão de propriedades leiteiras, 3,6 milhões de empregos gerados apenas pelo segmento da pecuária leiteira \cite{NettoGomes2021} e a exportação de carne bovina representa 6\% do \sigla{PIB}{Produto Interno Bruto} nacional \cite{EmbrapaCarneBovina2024}.

De acordo com \citeonline{Oliveira2014}, a base da pecuária e suas ramificações, não só do Brasil, mas do mundo, é formada por técnicas de reprodução animal assistida. Isso acontece porque essas técnicas são úteis tanto para aumentar consideravelmente a quantidade de cabeças de gado, quanto para aprimorar a qualidade genética dos animais. 

Nesse contexto, existe a \sigla{PIV}{Produção \textit{in vitro}} como uma importante técnica de reprodução assistida. Esse método permite que haja uma seleção das melhores células reprodutoras antes de serem enviadas para os laboratórios responsáveis por finalizar o procedimento. Essa seleção precisa ser precedida de uma classificação morfológica e funcional, que ocorre logo após a coleta das células de interesse \cite{MELLO}. Para realizar a classificação, é necessário um profissional da medicina veterinária para analisar as características de célula por célula. Portanto, nota-se como esse processo está dependente de um trabalho longo, manual e condicionado à profissionais capacitados.

Outra área de grande relevância para a atualidade é a \sigla{TI}{Tecnologia da Informação}. Um de seus princípios fundamentais consiste em atuar de forma integrada com outras áreas do conhecimento, visando automatizar processos ou aprimorar a eficiência de procedimentos tradicionalmente manuais. Tal contribuição é possível porque os sistemas computacionais são capazes de processar grandes volumes de dados em curto tempo e de forma isenta do enviesamento característico das análises humanas \cite{henrique2024processamento}

Através de tecnologias modernas de inteligência artificial, é possível desenvolver modelos de aprendizado de máquina que classifiquem automaticamente oócitos, as células reprodutoras femininas, coletados em anima. Com uma classificação aprimorada, é possível também melhorar a etapa de seleção dos oócitos que serão os enviados para os laboratórios de PIV. Dessa forma, o aprimoramento genético oriundo desse processo pode ocasionar uma grande melhoria na qualidade do gado brasileiro \cite{Mueller2022}.

Por fim, a automatização do reconhecimento e classificação de oócitos bovinos é realizada através da coleta de imagens dessas células reprodutoras em laboratório, pré-processamento de cada uma delas e treinamento de dois modelos diferentes: um para identificação e outro para classificação.


	
	\chapter{Objetivos}
	\label{chapter:objetivos}
	\section{Objetivo Geral}
Construir um modelo capaz de reconhecer e classificar oócitos bovinos quanto à sua morfologia e viabilidade funcional, com vistas ao processo de reprodução animal assistida. 


\section{Objetivos Específicos}
Com o desdobramento do objetivo geral, pretende-se atingir os seguintes objetivos específicos:

\begin{itemize}
	\item Aplicar técnicas de aprendizado de máquina nas tarefas e reconhecimento e classificação e oócitos;
	\item Estudar técnicas de processamento digital de imagens;
	\item Analisar a capacidade de generalização de métodos de\textit{ machine learning} aplicada a reprodução animal;
	\item Aplicar e avaliar métodos de Redes Neurais Convolucionais nas tarefas de reconhecimento e classificação de oócitos.
\end{itemize}




	
	%\chapter{Definição do Problema}
	%\label{chapter:definicao-do-problema}
	%\input{tex/definicao-do-problema}
	
	%\chapter{Hipótese}
	%\label{chapter:hipotese}
	%\input{tex/hipotese}
	
	\chapter{Justificativa}
	\label{chapter:justificativa}
	
A economia brasileira é fortemente marcada por atividades econômicas que envolvem bovinos. O Brasil detém o segundo maior rebanho do mundo, é o segundo maior produtor de carne bovina \cite{FAOSTAT2022c}, é o maior exportador e é o quinto maior produtor de leite \cite{FAO2019}. A criação de bovinos é muito importante para a geração de empregos, para a economia nacional e para a produção de alimentos altamente nutritivos para o Brasil e para o mundo \cite{teixeira2014trajetoria}.

Isso se deve ao fato de que o Brasil, por estar localizado em clima tropical, não pôde copiar as tecnologias de pecuária de países de clima temperado, como os Estados Unidos da América. Essa barreira impulsionou o avanço tecnológico e fez com que os brasileiros desenvolvessem suas próprias tecnologias, que acabaram servindo de modelo para outros países de clima tropical \cite{Camargo2022}.

As raças bovinas mais produtivas evoluíram e foram melhoradas geneticamente em países de clima temperado, inicialmente na Europa, posteriormente na América do Norte e em outras regiões de clima temperado. Como a maior parte do Brasil está localizada na faixa tropical, é necessário o desenvolvimento de animais adaptados à esta condição ambiental \cite{oosting2014development}, o que está sendo feito com enorme pioneirismo e competência por parte dos setores de produção de carne e de leite do país. Para acelerar este melhoramento genético, as técnicas de reprodução assistidas são imprescindíveis, como a inseminação artificial e a produção \textit{in vitro} de embriões \cite{sales2024evolution}.

A produção \textit{in vitro} possui várias etapas no seu funcionamento e uma delas é a classificação de células reprodutoras, como os gametas femininos nomeados de oócitos \cite{MELLO}. Atualmente, essa etapa é realizada de forma manual \cite{Shivaani2024}, algo que ocasiona em um grande número de horas de trabalho por parte dos profissionais da medicina veterinária que realizam esse serviço. 

Nesse contexto, surgem sistemas de informação que utilizam modelos inteligentes que pode aprimorar muito a eficiência de procedimentos já existentes e realizados por humanos. Dessa forma, percebe-se como há espaço para uma parceria entre pecuária e computação com o objetivo de agilizar e aprimorar a produção \textit{in vitro}.

Portanto, essa proposta se justifica pelo impacto social e econômico que ela pode produzir para a pecuária nacional.
	
	\chapter{Revisão Bibliográfica/Teórica}
	\label{chapter:revisao-bibliografica-teorica}
	Durante a evolução dos mamíferos, a necessidade de garantir a continuidade das espécies fez com que esses animais desenvolvessem e aprimorassem seus métodos reprodutivos. Essas reproduções acontecem através da união do gameta masculino, chamado de espermatozoide, com o gameta feminino, conhecido como oócito. Essa união origina um zigoto, que se divide uma série de vezes até gerar um embrião \cite{HAFEZ}.  

No entanto, a atualidade, junto com sua tecnologia, possui estratégias que facilitam e aprimoram todo esse processo, como métodos modernos de reprodução assistida. Dentre eles, existe a produção \textit{in vitro} de embriões (PIV), que garante uma série de benefícios para a reprodução.


\section{Produção \textit{in vitro}}
O processo de desenvolvimento dessa técnica originou-se entre 1877 e 1899. Nessa época, pesquisadores buscavam estabelecer estratégias que permitissem a manipulação de embriões. Isso resultou na primeira visualização da fecundação de um oócito de estrela do mar que, por sua vez, originou um zigoto. Com o passar das décadas, mais especificamente em 1959, houve o nascimento do primeiro coelho gerado a partir da técnica de PIV  \cite{MELLO}.

De acordo com \citeonline{MELLO}, a PIV é uma biotécnica cujo objetivo é explorar o potencial genético de fêmeas bovinas. Ela é estratificada nas seguintes etapas: coleta, maturação, fecundação e cultivo \textit{in vitro}.  Além do mais, segundo \citeonline{GOUVEIA}, a PIV é importante por conta de sua importância para estudos biotecnológicos e fundamentos comerciais. 

Conforme a \citeonline{EMBRAPA}, a PIV apresenta as seguintes vantagens:


\begin{adjustwidth}{4cm}{0cm} % Aumenta o recuo da margem esquerda para 4cm
	\fontsize{10}{12}\selectfont % Ajusta o tamanho do texto para 10 pontos
	Possibilita a utilização de bezerras pré-púberes, vacas em início de gestação, vacas com subfertilidade adquirida, vacas senis e vacas mortas acidentalmente; produção de cerca de 36 bezerros por ano a partir de uma única fêmea; avanço na multiplicação de fêmeas bovinas de interesse para a produção animal e para a conservação de raças de animais domésticos ameaçadas de extinção; facilita o uso e aprimoramento de técnicas avançadas de reprodução animal, como: clonagem; injeção intracitoplasmática de espermatozoides; e transgenia; permite otimizar o uso de sêmen de reprodutores alto valor genético e de sêmen sexado; permite a produção de embriões com grau de sangue e sexo definidos para atender a programas específicos de produção (leite e carne), em larga escala e com menor custo.
\end{adjustwidth}

Entre as etapas de coleta e maturação, precisa existir uma fase de classificação dos oócitos coletados
\cite{PENITENTE}. Isso é importante porque algumas células não possuem as propriedades necessárias para seguir em frente no processo e algumas são melhores que outras. 



\subsection{Classificação de Oócitos}
Os oócitos coletados precisam ser divididos em quatro categorias, de acordo com as propriedades das células do cumulus e do ooplasma \cite{PENITENTE}. Esse método de classificação foi desenvolvido por \citeonline{LEIBFRIED}.

De acordo com \citeonline{SPMRcumulus}, as células do cumulus, um dos fatores de classificação mencionados, envolvem o oócito e desempenham papéis essenciais como garantir a comunicação bidirecional e proteger o oócito. A estrutura dessas células está ilustrada na Figura \ref{fig:oocito}.

Além do mais, segundo \citeonline{ELSEVIERcitoplasma}, o ooplasma, que é o segundo fator de classificação, é apenas o citoplasma do oócito. Ele é um líquido viscoso, localizado dentro da membrana citoplasmática e armazena toda a estrutura interna da célula \cite{ELSEVIERcitoplasma}. Seu posicionamento está representada na Figura \ref{fig:oocito}.

\begin{figure}[H]
	\centering
	\caption{Representação dos componentes de um oócito}
	\includegraphics[width=0.6\textwidth]{images/oocitos.png}
	\caption*{\textbf{Fonte: Elaboração própria.}} 
	\label{fig:oocito}
\end{figure}

Por fim, de acordo com \citeonline{PENITENTE}, os oócitos viáveis são os de classificação I a III, enquanto os de classificação IV são desconsiderados. 

\subsubsection{Classificação I}
Nesta categoria, as células do cumulus ou células da granulosa precisam formar, no mínimo, mais de três camadas \cite{PENITENTE}. Enquanto isso, o ooplasma precisa possuir coloração marrom com granulações finas e homogêneas \cite{PENITENTE}, conforme ilustrado na Figura \ref{fig:oocitoexemplo}

\begin{figure}[H]
	\centering
	\caption{Representação de um oócito com mais de três camadas de células da granulosa}
	\includegraphics[width=0.6\textwidth]{images/oocito/figura2.png}
	\caption*{\textbf{Fonte: Elaboração própria.}} 
	\label{fig:oocitoexemplo}
\end{figure}

\subsubsection{Classificação II}
Nesta classificação, os oócitos possuem menos de 3 camadas de células da granulosa, enquanto o ooplasma possui granulações distribuídas heterogeneamente \cite{PENITENTE}, conforme ilustrado na Figura \ref{fig:oocito2}. 

\begin{figure}[H]
	\centering
	\caption{Representação de um oócito com menos de três camadas de células da granulosa e com ooplasma heterogêneo}
	\includegraphics[width=0.6\textwidth]{images/oocito/figura3.png}
	\caption*{\textbf{Fonte: Elaboração própria.}} 
	\label{fig:oocito2}
\end{figure}

\subsubsection{Classificação III}
Nesta classificação, os oócitos possuem camadas de células da granulosa. No entanto, há um espaço entre a membrana celular e a zona pelúcida \cite{PENITENTE}, conforme ilustrado na figura \ref{fig:oocito5}. 

\begin{figure}[H]
	\centering
	\caption{Representação dos componentes de um oócito com espaço entre as células da granulosa e a zona pelúcida}
	\includegraphics[width=0.6\textwidth]{images/oocito/figura04.png}
	\caption*{\textbf{Fonte: Elaboração própria.}} 
	\label{fig:oocito5}
\end{figure}

\subsubsection{Classificação IV}
Nesta classificação, os oócitos não possuem células da granulosa ou possuem citoplasma com cor e granulação fora do normal \cite{PENITENTE},  conforme a Figura \ref{fig:oocito6}. 

\begin{figure}[H]
	\centering
	\caption{Representação dos componentes de um oócito sem camadas de células da granulosa}
	\includegraphics[width=0.6\textwidth]{images/oocito/figura5.png}
	\caption*{\textbf{Fonte: Elaboração própria.}} 
	\label{fig:oocito6}
\end{figure}



\section{Visão Computacional}

As ciências da computação possuem um ramo de estudo chamado de visão computacional \cite{BALLARD}. Essa área visa aprimorar as capacidades computacionais  de compreensão e interpretação de conteúdos visuais \cite{BALLARD}. Isso permite a criação de sistemas robustos e inteligentes que podem ser utilizados em diversos campos do conhecimento humano. 

O processamento de imagens é o cerne da visão computacional. Ele tem como objetivo coletar das imagens os tipos de dados definidos como parâmetros, ou seja, os que serão utilizados pelo sistema. Finalizada essa etapa, acontece o procedimento de classificação de entidades presentes na imagem \cite{JAHNE}. Segundo \cite{JAHNE}, esse processamento pode ser estratificado em uma série de passos até o resultado final ser alcançado. Alguns desses passos são dispensáveis, mas outros não.

De acordo com \citeonline{JAHNE}, o primeiro passo deve ser o ato de aquisição de imagem, algo que pode ser feito através de câmeras digitais ou analógicas. No entanto, essa última forma torna obrigatório o processo de digitalização da imagem para que se torne apropriada para o processo. Logo em seguida, efetua-se o pré-processamento cujo objetivo é eliminar eventuais imperfeições das imagens coletadas. Com essas etapas finalizadas, o reconhecimento de objetos pode ser realizado através da extração de características.

\subsection{Processamento Digital de Imagens}

A explicação sobre o que é uma imagem acontece através de uma função bidimensional, cuja representação é f(x,y). Nessa função, x e y são coordenadas em um plano e a amplitude de f em qualquer ponto desse plano é conhecida como intensidade. Caso as quantidades de x, y e os valores de intensidade de f sejam finitas e discretas, a imagem será definida como digital. Além disso, a composição de uma imagem digital é um número finito de elementos cujas localizações e valores são particulares para um pixel \cite{GONZALES}.

Segundo \citeonline{GONZALES}, existem passos fundamentais que devem ser seguidos durante o processamento digital de imagens, sendo elas:


\begin{itemize}
	\item \textbf{Aquisição de Imagem}: é a primeira etapa do processo e envolve o recebimento de imagens em formato digital. Também pode haver a necessidade de um pré-processamento para adequar a imagem ao sistema que a analisará;
	
	\item \textbf{Filtragem e Realce de Imagens}: esta etapa consiste em manipular uma imagem para que ela se torne mais apropriada para o processamento de um sistema específico. É importante frisar que esta fase varia de acordo com as necessidades de cada contexto;
	
	\item \textbf{Restauração de Imagens}: esta área também trata do aprimoramento visual de imagens. No entanto, há uma objetividade maior aqui, tendo em vista que seus métodos de aplicação são oriundos da matemática ou estatística;
	
	\item \textbf{Processamento de Imagens Coloridas}: a cor pode ser um importante fator descritivo de determinado objeto, ou seja, o sistema que puder fazer uso desse artifício pode vir a ser mais eficiente no processo de identificação;
	
	\item \textbf{Extração de Características}: também conhecida como processamento morfológico, essa etapa é responsável pela extração de componentes cuja importância é vital para a representação dos atributos da imagem;
	
	
	\item \textbf{Reconhecimento de Objetos}: da mesma forma que um humano é capaz de identificar objetos com o sentido da visão, uma máquina também pode ser capaz de realizar essa mesma identificação. No entanto, isso é apenas possível por meio da codificação de todas as etapas do processamento de imagens \cite{JAIN};
\end{itemize}


Dito isso, é importante elucidar sobre um aspecto crucial para o desenvolvimento da acurácia do reconhecimento, que é o \textit{dataset}. Essa tecnologia consiste em um grande aglomerado de dados que servirão para a realização de análises. Nessa coleção de dados, existem alguns que serão utilizados pelo sistema como uma referência positiva, para que seja possível deliberar se determinado objeto de análise poderá ser classificado como objeto de interesse. Enquanto isso, as referências negativas servirão para que o sistema possa saber quais aspectos ignorar no processo de análise \cite{JAIN}.

As características de uma imagem deverão ser avaliadas individualmente assim que a fase de reconhecimento for iniciada. Primeiramente, o sistema irá verificar se a imagem possui certo grau de características candidatas a serem parte do objeto de interesse. Se esse resultado for positivo, a imagem será comparada com as do \textit{dataset}. Dessa forma, se as comparações forem válidas, a imagem será classificada como um objeto de interesse \cite{JAIN}.

\subsection{OpenCV}

De acordo com a documentação \citeonline{OPENCV2024}, a OpenCV é uma sigla cujo significado é biblioteca de visão computacional e de código aberto. Essa ferramenta é gratuita, algo que a torna ideal para o desenvolvimento de projetos acadêmicos. Além do mais, ela é amplamente utilizada em \textit{softwares} que aplicam o aprendizado de máquina. 

A biblioteca OpenCV consiste em cinco aspectos fundamentais \cite{BRADSKI2008}:

\begin{itemize}
	\item \textbf{CV}: esse componente possui o objetivo de realizar o processamento de imagens através da visão computacional;
	
	\item \textbf{MLL}: esse componente é responsável pelo aprendizado de máquina, ou seja, ele irá executar modelos treinados e manipular as saídas por meio dos parâmetros definidos;
	
	\item \textbf{HighGUI}: esse componente trata das questões relacionadas a imagens e vídeos. Além do mais, ele possui uma funcionalidade de geração de janelas de visualzação das mídias que serão adicionadas ao código;
	
	\item \textbf{CXCore}: esse componente possui a responsabilidade de englobar todos os aspectos citados em uma apresentação simples e sucinta para que o desenvolvedor possa ter uma compreensão mais adequada;
	
	\item \textbf{CvAux}: permite que o programador utilize funções de reconhecimento facial e reconhecimento de gestos.
	
	
	
\end{itemize}



\section{Inteligência Artificial}
A \sigla{IA}{Inteligência Artificial} é um ramo das ciências da computação cujo objetivo é desenvolver sistemas de computadores capazes de pensar e agir como seres humanos. Essa tecnologia é capaz de desenvolver uma série de ferramentas capazes de aprimorar a produção nas mais diversas áreas do trabalho humano, como por exemplo: extração e processamento de dados, automação de processos, etc \cite{CARVALHO}. 

Dessa forma, percebe-se a evolução que a IA é capaz de realizar diante de algoritmos convencionais tendo em vista que eles apenas leem e executam as linhas sequenciais de um código, ou seja, não são capazes de aprender de nenhuma forma. Portanto, sua aplicabilidade é extremamente limitada \cite{SICHMAN}.

Anteriormamente à IA, o ser humano era o único responsável pelo aprendizado e desenvolvimento de soluções para problemas complexos. No entanto, a inteligência artificial surge como uma alternativa para essa realidade \cite{SICHMAN}.

\subsection{Aprendizado de Máquina}

A Inteligência Artificial (IA) possui uma ramificação nomeada de \sigla{AM}{Aprendizado de Máquina} cujo foco é utilizar grandes quantidades de dados para criar e treinar modelos. Esses modelos, por sua vez, simulam o comportamento humano de resolução de problemas complexos que necessitam de interpretações para serem solucionados, ou seja, um algoritmo convencional não tem como resolvê-los \cite{SMOLA}. 

De acordo com \citeonline{DEISENROTH}, o cerne do aprendizado de máquina gira em torno de dados, modelos e processamento. Os dados possuem uma característica basilar para o AM, já que são eles os responsáveis pela etapa crucial de treinamento. Os modelos, por sua vez, são responsáveis por lidar com situações não apresentadas no \textit{dataset}, ou seja, aqui há a etapa de aprendizado. 

Além do mais, segundo \citeonline{agrawal2020}, o AM possui dois tipos de variáveis vitais para o aprendizado do modelo: os hiperparâmetros e os parâmetros. O primeiro é de caráter manual, pois é necessário determiná-los previamente em relação ao treinamento. O segundo é volátil, uma vez que o próprio modelo é incumbido da responsabilidade de alterá-los à medida que novas informações vão sendo alimentadas.

\subsection{Redes Neurais Artificiais}
De acordo com \citeonline{HAYKIN}, o \sigla{RNA}{Redes Neurais Artificiais} são estudadas e aplicadas pelo fato de que o cérebro humano é capaz de processar informações de forma muito mais eficiente do que um computador por causa das propriedades de complexidade, não-linearidade e paralelismo presentes no cérebro humano. Portanto, um sistema de aprendizado possui muito a ganhar caso consiga reproduzir tal estrutura. 

Além do mais, as habilidade de generalização é outro fator que aumenta consideravelmente o poder computacional de uma RNA. A generalização é a capacidade de fornecer saídas adequadas para entradas que não estavam presente no \textit{dataset} utilizado para treinar o sistema \cite{HAYKIN}. 

No entanto, as redes neurais não são capazes de fornecer resultados trabalhando isoladamente. Elas precisam ser integradas em um sistema maior. Isso acontece da seguinte forma: um problema complexo e dividido em várias tarefas menores e simples, e então grupos de subtarefas serão entregues para redes neurais cuja especialização mais se assemelha às necessidades da tarefa \cite{HAYKIN}.

\subsubsection{Redes Neurais Profundas}
Embora Redes Neurais Profundas (RNP) sejam semelhantes à Redes Neurais Artificiais, há um acréscimo de camadas empilhadas que as diferencia, como ilustrado na figura \ref{fig:ia}.

\begin{figure}[H]
	\centering
	\caption{Rede Neural Simples e Rede Neural Profunda}
	\includegraphics[width=0.8\textwidth]{images/ia/figura6.png}
	\caption*{\textbf{Fonte: Adaptado de \citeonline{BOHANI}.}} 
	\label{fig:ia}
\end{figure}

Em relação à essas camadas adicionais, quanto mais delas existirem na RNP, mais complexo o sistema será, mais recursos computacionais precisarão ser utilizados e mais tempo será necessário para treinar o modelo \cite{SUBASI202091}. 

De acordo com \citeonline{MOHANASUNDARAM2019139}, cada camada irá trabalhar com características diferentes baseadas na saída da camada anterior. Isso significa que, quanto mais se avançar na rede, mais complexas serão as características que as camadas reconhecerão tendo em vista que elas se rearranjam e aprendem com as características das camadas anteriores. A vantagem desse processo é que as redes conseguem modelar relações não lineares e complexas, algo que permite o tratamento de dados não rotulados e não estruturados. 

Portanto, nota-se como as redes neurais profundas são capazes de coletar dados brutos, não rotulados, desestruturados e ainda conseguir agrupá-los e realizar o processamento deles \cite{MOHANASUNDARAM2019139}. 


\subsubsection{Redes Neurais Convolucionais}
De acordo com \citeonline{VARGAS2016}, uma \sigla{RNC}{Rede Neural Convolucional} nacional é uma variação das Redes Neurais Profundas cujo funcionamento é inspirado no processamento biológico de dados visuais. Ademais, semelhantemente a outros métodos da visão computacional, uma RNC também aplica filtros em dados visuais ao mesmo tempo que armazena os relacionamentos de vizinhança entre os \textit{pixels} da imagem com o decorrer do processamento. Segundo \citeonline{ZHANG2018146}, esse é o modelo mais utilizado de RNP em aprendizado de características para classificação e reconhecimento de imagens em grande escala.

O funcionamento de uma RNC é dividido em três etapas principais, que são: a convolução, subamostragem e classificação. \cite{ZHANG2018146}. A convolução, por sua vez, necessita de componentes básicos para funcionar adequadamente, que são: dados de entrada, filtros e mapas de características.

Segundo \citeonline{Alves2018}, as entradas da convolução são consideradas como matrizes tridimensionais cujas duas primeiras dimensões são a altura e a largura da imagem, enquanto a profundidade da terceira dimensão é definida por quantos canais de cores a imagem tem. Por exemplo: se determinada imagem seguir o padrão RGB, a profundidade da terceira dimensão será de três canais. 

Já o filtro, é uma matriz bidimensional com altura e largura menores que as da da entrada cuja função é percorrer a imagem para assimilar as principais características. Ela percorre a imagem de acordo com uma \textit{stride}, que são os saltos feitos pelo filtro para se locomover. Quando todo o trajeto tiver sido percorrido, é gerado um mapa de características, que será a primeira\textit{ hidden layer} \cite{Alves2018}. Dessa forma, à cada convolução, a RNC amplia sua complexidade e identifica trechos maiores da mídia visual que foi fornecida como entrada. Isso significa que as primeiras camadas se concentram em aspectos simples, enquanto as camadas convolucionais finais irão reconhecer características mais complexas \cite{IBM2024}.

Logo em seguida, ocorre a subamostragem (\textit{subsampling}). Ela irá simplificar a saída da camada anterior através de uma matriz bidimensional que irá percorrer toda a saída. À media que essa matriz se locomove, ela escolhe o maior valor dos trechos percorridos e vai gerando outra matriz formada apenas pelos maiores valores. Isso resulta em uma generalização que irá consumir menos recursos computacionais nas etapas seguintes e impede o \textit{overfitting}, que acontece quando um modelo de aprendizado de máquina se adapta excelentemente a um conjunto de dados, mas não funciona adequadamente com dados novos. 

Por fim, a classificação ocorre na camada totalmente conectada através da combinação das características assimiladas pelas camadas anteriores \cite{IBM2024}.

Para ilustrar melhor esse procedimento, a Figura \ref{fig:rnc} abaixo mostra a estrutura de uma das primeiras Redes Neurais Convolucionais:

\begin{figure}[H]
	\centering
	\caption{Arquitetura da LeNet-5, uma Rede Neural Convolucional}
	\includegraphics[width=0.9\textwidth]{images/ia/rnc.png}
	\caption*{\textbf{Fonte: \citeonline{lecun1998gradient}.}} 
	\label{fig:rnc}
\end{figure}

Como pode ser visto na imagem, uma matriz com vários valores percorre a imagem de entrada e quais serão esses valores depende exclusivamente do tipo de resultado desejado. A partir disso, os valores da matriz serão multiplicados por cada número que representa o \textit{pixel}, os resultados das multiplicações serão somados e a somatória de todo esse processo ocupará um pixel da imagem de saída. Isso é repetido até a imagem inteira ser percorrida e um \textit{feature map} ser gerado \cite{kumar2025basiccnn}. 

Com isso, o procedimento de \textit{subsampling} começa 
para reduzir as dimensões do \textit{feature map} ao mesmo tempo que valores relevantes são mantidos com o objetivo de generalizar as informações importantes para o aprendizado do modelo, algo que impede o problema de \textit{overfitting}, que ocorre quando um modelo se adapta demais a uma imagem específica e não funciona bem com outros objetos de interesse da mesma classificação \cite{kumar2025basiccnn}.


Logo em seguida, as etapas de \textit{full conection} realizam a classificação com base nas características relevantes extraídas. Esse procedimento é feito da seguinte forma: as matrizes de cada \textit{feature map} são achatadas para se tornarem unidimensionais e são transformadas em um único grande vetor \cite{kumar2025basiccnn}. 

Os valores desse vetor são processados para que as características sejam aprendidas e associadas a probabilidades para cada classe de interesse \cite{kumar2025basiccnn}. 



\subsubsection{Segmentação de Instância}

De acordo com \citeonline{abdulla2018}, segmentação de instância é o processo de identificar o contorno de um objeto no nível dos \textit{pixels}. 

Primeiramente, ocorre a classificação, ou seja, percebe-se que objetos de determinada classe estão presentes na imagem. Logo após, ocorre a segmentação de semântica cujo objetivo é identificar todos os \textit{pixels} que pertencem à classe de interesse. Depois, a detecção de objetos nota a presença de sete instâncias, mas ainda não consegue diferenciar os \textit{pixels} que se sobrepõem. Por fim, a segmentação de instância é capaz de discernir os contornos de cada objeto, ou seja, o problema de sobreposição foi solucionado. Esse procedimento está descrito na Figura \ref{fig:segmenta}.


\begin{figure}[H]
	\centering
	\caption{Etapas da Segmentação de Instância}
	\includegraphics[width=0.9\textwidth]{images/ia/segmenta.png}
	\caption*{\textbf{Fonte: \citeonline{abdulla2018}{}.}} 
	\label{fig:segmenta}
\end{figure}

\subsubsection{R-CNN}

De acordo com \citeonline{Yenidun2025}, a CNN utiliza uma matriz que percorre uma imagem inteira até encontrar um objeto de interesse. Tendo em vista que esse processo é pouco eficiente e exige muitos recursos computacionais, a \textit{Region-based Convolutional Neural Networks} (R-CNN) aparece como uma solução para otimizar esse busca a partir de determinações das regiões mais prováveis que o objeto irá aparecer. 

Dessa forma, a R-CNN possui 4 principais etapas:


\begin{itemize}
	\item \textbf{Proposta de Região}: um conjunto de propostas de regiões é gerado cada uma delas funciona como uma matriz que pode ou não conter objetos de interesse ;
	
	\item \textbf{Extração de Características}: cada região é proposta é redimensionada para um tamanho fixo porque um classificador de rede neural necessita de dimensões padronizadas;
	
	\item \textbf{Classificação}: verificar se há objetos de interesse dentro de cada região e classificá-los de acordo com suas classes caso existam;
	
	\item \textbf{Regressão de caixa delimitadora}: as matrizes que possuírem objetos de interesse dentro delas serão redimensionadas para que se enquadrem adequadamente aos limites dos objetos.
\end{itemize}




\subsubsection{Mask R-CNN}
O Mask R-CNN, por sua vez, é uma evolução do R-CNN porque possui duas características adicionais: uma rede treinável de proposta de região e segmentação de instância \cite{Yenidun2025}. 

Dessa forma, a rede aprenderá quais são as regiões mais prováveis de conter objetos de interesse, enquanto a segmentação de instância para definir os contornos dos objetos ao nível do pixel, algo que facilita a compreensão de objetos que se sobrepõem \cite{Yenidun2025}. 

\subsubsection{\textit{COCOEvaluator}}
A avaliação de modelos de aprendizado de máquina é um procedimento que transforma pares de entradas e saídas em métricas agregadas. Dessa forma, é possível quantificar a eficiência da execução de um modelo. Um método que realiza esse tipo de tarefa é o \textit{COCOEvaluator} porque ele é capaz de calcular a precisão média de caixas delimitadoras e segmentações de instâncias \cite{detectron2}. 

Para conseguir realizar esses cálculos, o COCOEvaluator precisa trabalhar com determinadas medidas. A primeira delas é a \textit{Insertection over Union} (Iou), que é a base para todas as medidas, serve para calcular a proporção entre a área de intersecção e a área da união entre o espaço previsto pelo modelo e a anotação real feita no pré-processamento. Portanto, o resultado ótimo para essa proporção é 1.0 e quanto mais perto desse valor, melhor \citeonline{rosebrock2016iou}. Isso acontece porque um resultado de 1.0 indica que a área prevista está exatamente onde a anotação foi feita. Essa fórmula pode ser visualizada na Figura \ref{fig:formula}.


\begin{figure}[H]
	\centering
	\caption{Fórmula do IoU}
	\includegraphics[width=0.8\textwidth]{images/ia/formula.png}
	\caption*{\textbf{Fonte: \citeonline{rosebrock2016iou}.}} 
	\label{fig:formula}
\end{figure}

Além do mais, as caixas delimitadoras definem o quão bem o modelo consegue desenhar um retângulo ao redor do oócito, enquanto a segmentação verifica o quão bem o modelo desenha uma máscara nos contornos exatos do oócito, ou seja, \textit{pixel} por \textit{pixel}. E, por fim, \textit{Average Precision} (AP), é simplesmente o resultado médio realizado a partir de vários valores diferentes de IoU. Primeiramente, é preciso explicar os termos da tabela para compreênde-la adequadamente \citeonline{rosebrock2016iou}.
	
	\chapter{Metodologia}
	\label{chapter:metodologia}
	
%VERIFICAR SE HÁ UMA PADRONIZAÇÃO DE DISTANCIA.

%

Neste capítulo, serão apresentadas todas as ferramentas e os procedimentos utilizados para a construção do modelo computacional de reconhecimento e classificação oócitos bovinos.

Primeiramente, os oócitos foram visualizados através de instrumentos de aproximação por conta de seu tamanho microscópico e as suas imagens foram capturadas e armazenadas. Em seguida, as imagens captadas são pré-processadas para a retirada de ruídos e padronização de camadas; processadas em uma rede neural para reconhecer as regiões onde se encontram os oócitos; construção do modelo para classificá-los; e, finalmente, um modelo final que une essas duas funcionalidades. Esse processo pode ser visualizado na figura \ref{fig:processo} e será descrito com mais detalhes nas seções seguintes.

\begin{figure}[H]
	\centering
	\caption{Demonstração das Etapas de Construção do Modelo}
	\includegraphics[width=0.9\textwidth]{images/ia/processo6.png}
	\caption*{\textbf{Fonte: Elaboração própria}}
	\label{fig:processo}
\end{figure}

\section{Visualização e Coleta}

Os oócitos utilizadas neste trabalho foram fornecidas pela empresa parceira BioInova \cite{bioinnova2024}. Ela possui experiência com produções \textit{in vitro} desde 2019 e, por conta disso, possui um vasto \textit{dataset} de oócitos que será necessário para realizar o processo de aprendizado do sistema .

A visualização e captura das imagens são feitas utilizando um instrumento de aproximação microscópico. Ele é demonstrado na Figura \ref{fig:lupa}

\begin{figure}[H]
	\centering
	\caption{Demonstração das componentes de um Microscópio Estereoscópio}
	\includegraphics[width=0.8\textwidth]{images/ferramentas/lupa.png}
	\caption*{\textbf{Fonte: \citeonline{gomes2005}.}} 
	\label{fig:lupa}
\end{figure}

Ele é equipado com uma câmera, o que permite a coletada das imagens de maneira apropriada. Essa é composta por um conjunto óptico de três câmeras, disponíveis num celular Samsung Galaxy s20 fe, que segundo o fabricante, é composta  12, 64 e 12 megapixels (MP) e aberturas de f1.8, f2.0 e f2.2, respectivamente (SAMSUNG, 2024).

\section{Pré-processamento}

Nessa seção, todas as etapas necessárias para preparar as condições mínicas necessárias para o desenvolvimento do modelo de identificação e classificação de oócitos serão abordadas. 

\subsection{OpenCV}
O pré-processamento consiste em corrigir aspectos ruidosos nas imagens, tais como: sujeiras, áreas ofuscadas, dentre outros. Essas correções podem ser radiométricas ou geométricas \cite{CHAKI2019} e são realizadas através da biblioteca OpenCV e seus métodos.

Algumas imagens passaram por uma correção radiométrica \cite{CHAKI2019}. Esse tipo de problema surge quando há um mal posicionamento do objeto capturado diante da luz ou falta de calibração adequada nos sensores. Isso faz com que alguns \textit{pixels} da imagem não sejam constituídos, ou seja, será necessário reconstituí-los de forma artificial através dos \textit{pixels} mais próximos como referência \cite{CHAKI2019}. Para tratar esse tipo de problema, os fundamentos utilizados são equalização de histograma e conversão de espaço de cores\ \cite{his} . 

Houve também casos em que movimentos durante o processo de captura e/ou lentes distorcidas no dispositivo de captação ocasionam um posicionamento distorcido de determinados \textit{pixels} da imagem \cite{CHAKI2019}. Para resolver isso, foi necessário reposicionar os \textit{pixels} que foram desvirtuados através das técnicas de transformações geométrica da OpenCV, que são: correção de distorção de lente, transformação de perspectiva, e transformação afim \cite{geo}. 

Para ajustes na iluminação, utilizou-se filtros de gradiente e equalizações de histograma, que servem para calibrar a regularidade de intensidade de \textit{pixels} \cite{KRIG2014}. 

Por fim, houve a necessidade de realizar ajustes relacionados ao foco de determinadas regiões da imagem captada. Por exemplo: se alguma região de interesse estiver desfocada ou se alguma região irrelevante estiver com muito foco, ambas essas situações precisarão ser reajustadas para que o processamento ocorra adequadamente \cite{KRIG2014}. 

\subsection{LabelMe}
O \textit{LabelMe} também apresentou-se como uma eficiente tecnologia para pré-processamento tendo em vista sua praticidade em carregar \textit{datasets}, opção para desenhar os círculos nas imagens e geração dos JSONs que contém os dados pré-processados \cite{russell2008labelme}. 


Um exemplo de imagem pré-processada pode ser visualizado na Figura \ref{fig:imagempre}.


\begin{figure}[H]
	\centering
	\caption{Imagem pré-processada}
	\includegraphics[width=0.8\textwidth]{images/ia/imagempre.png}
	\caption*{\textbf{Fonte: Elaboração Própria.}}
	\label{fig:imagempre}
\end{figure}


\section{Construção do Modelo de Identificação}
Com o conjunto de imagens completamente pré-processado, foi possível utilizar o Detectron2, que de acordo com \cite{Lad2024Detectron2}, é um \textit{framework} para diversos algoritmos de detecção de objetos e, entre eles, encontra-se o Mask R-CNN.  

Logo após, um modelo de aprendizado de máquina para identificar oócitos bovinos foi gerado com o Detectron2. A arquitetura desse modelo está representada na Figura \ref{fig:arquitetura}

\begin{figure}[H]
	\centering
	\caption{Arquitetura do Modelo de Aprendizado de Máquina para Identificação de Oócitos}
	\includegraphics[width=0.8\textwidth]{images/ia/arquitetura2.png}
	\caption*{\textbf{Fonte: Elaboração Própria.}} 
	\label{fig:arquitetura}
\end{figure}

Percebe-se uma arquitetura baseada no \textit{Mask R-CNN} utilizado pelo \textit{Detectron2} e adaptada especificamente para detecção e segmentação de oócitos bovinos.

Essa rede é composta por três módulos principais:


\begin{itemize}
	\item \textbf{Backbone (ResNet + FPN)}: as imagens dos oócitos foram processadas e suas características, extraídas. A \textit{Feature Pyramid Network} (FPN) foi responsável por combinar informações em múltiplas escalas (fpnlateral2 – 5) com convoluções de 256 canais, algo que permitiu detectar oócitos de diferentes tamanhos e contrastes;
	
	\item \textbf{\textit{Region Proposal Network} (RPN)}: aqui, regiões de interesse foram geradas através de probabiliades de cada região conter um objeto de interesse;
	
	\item \textbf{ROI Heads (Box Head e Mask Head)}: as propostas geradas na etapa anterior foram refinadas através de classificações e ajustes das caixas delimitadoras dos oócitos.
	
\end{itemize}

Logo após, houve a construção do modelo de identificação com os conjuntos de nome \textit{oocitosTrain} e \textit{oocitosVal} configurados para 1 classe (oócito) e com 1000 iterações.

Com o modelo para identificar oócitos treinado, tornou-se possível desenvolver a etapa de classificação. Pode-se averiguar o funcionamento de modelo de identificação na Figura \ref{fig:entradaEsaida}

\begin{figure}[H]
	\centering
	\caption{Imagem de Entrada e Saída no Modelo de Identificação de Oócitos}
	\includegraphics[width=0.8\textwidth]{images/ia/saida2.png}
	\caption*{\textbf{Fonte: Elaboração Própria.}} 
	\label{fig:entradaEsaida}
\end{figure}

Para que esse modelo pudesse ser treinamento corretamente, foi utilizada uma proporção de 80\% da base de dados para treinamento e 20\% para validação, que é uma divisão padrão.



Após a finalização dessa etapa, tornou-se necessário avaliar a eficiência do modelo de identificação. Com esse intuito, o Detectron2 utiliza o método \textit{COCOEvaluator}.


\section{Construção do Modelo de Classificação}


 \section{Modelo Final}



	
	%\chapter{Resultados Esperados}
	%\label{chapter:resultados-esperados}
	%\input{tex/resultados-esperados}
	
	%\chapter{Cronograma}
	%\label{chapter:cronograma}
	%\input{tex/cronograma}
	
	\chapter{Resultados e Discussão}
	\label{chapter:resultados-discussao}
	Este tópico abordará os resultados obtidos e a discussão sobre eles acerca dos modelos de aprendizado de máquina para identificação e classificação de oócitos bovinos.
\section{Resultados do Modelo de Identificação}


A \autoref{tab:metricas_ap} demonstra os resultados obtidos pelo detectron2 através do COCO Evaluator, métrica padrão dessa rede.

% Definições de cores (certifique-se de que estão no PREÂMBULO ou antes da tabela)
\definecolor{mediumgray}{gray}{0.75}  % Cinza mediano
\definecolor{lightgray}{gray}{0.9}    % Cinza claro

\begin{table}[!htbp]
	\footnotesize
	\centering
	\caption{Métricas de Precisão Média (\textit{Average Precision} - AP) do COCOEvaluator.}
	
	% As colunas l X X X definem a estrutura da tabela
	\begin{tabularx}{\textwidth}{l X X X}
		\hline
		\rowcolor{mediumgray}
		\textbf{Métrica} & \textbf{Descrição} & \textbf{Caixas Delimitadoras} & \textbf{Segmentação} \\ 
		\hline
		
		\rowcolor{lightgray}
		$\text{AP}$ & AP Principal ($\text{IoU}=0.50$ a $0.95$, média) & $\mathbf{80.974\%}$ & $\mathbf{78.405\%}$ \\
		
		\rowcolor{white}
		$\text{AP}^{50}$ & AP com $\text{IoU} \ge 0.50$ (Detecção "fácil") & $97.783\%$ & $97.783\%$ \\
		
		\rowcolor{lightgray}
		$\text{AP}^{75}$ & AP com $\text{IoU} \ge 0.75$ (Detecção "rigorosa") & $96.033\%$ & $94.945\%$ \\
		
		
		\rowcolor{white}
		$\text{AP}_M$ & AP para objetos médios & $81.938\%$ & $79.493\%$ \\
		
		\rowcolor{lightgray}
		$\text{AP}_L$ & AP para objetos grandes & $81.022\%$ & $78.314\%$ \\
		\hline
	\end{tabularx}
	\caption*{Fonte: Elaboração Própria.}
	\label{tab:metricas_ap}
\end{table}

Dessa forma, pode-se destrinchar os elementos da tabela:

\begin{itemize}
	\item \textbf{AP (IoU=0.50 a 0.95, média)}: Demonstra uma alta precisão média do modelo (>75\%). A Detecção (80.974\%) é ligeiramente maior que a Segmentação (78.405\%);
	
	\item \textbf{{$\text{AP}^{50}$} (IoU$\geq$0.50)}: Há uma excelente localização porque o modelo consegue identificar praticamente todos os objetos corretamente nos dois processos com um IoU moderado;
	
	\item \textbf{{$\text{AP}^{75}$} (IoU$\geq$0.75)}: Mesmo com um limite consideravelmente mais rigoroso, a delimitação as caixas e dos contornos continua excelente e precisa;
	
	\item \textbf{{$\text{AP}_{M}$} (Objetos Médios)}: Há uma precisão adequada para ambos os processos quando se trata de oócitos de tamanha médio;
	
	\item \textbf{{$\text{AP}_{L}$} (Objetos Grandes)}: Apesar da precisão continuar boa, há uma pequena queda em comparação à oócitos médios;
\end{itemize}

O modelo de identificação apresentou uma eficiência excelente em seu desempenho por conta de sua alta precisão média mesmo nas métricas mais rigorosas. Além do mais, ele também é altamente idôneo tanto para as caixas delimitadoras quanto para as segmentações dos oócitos.

No entanto, uma comparação entre os valores revela uma distinção: a delimitação da caixa é ligeiramente mais precisa do que a tarefa de delimitação do contorno, apesar das duas possuírem bons resultados. Ademais, há um ótimo desempenho para identificação de objetos médios, enquanto há uma pequena queda para os objetos grandes. 

Por fim, esse modelo é um sucesso técnico tendo em vista que proporciona resultados com alta confiabilidade para ambos os principais processos de detecção e segmentação. Sua robustez garante uma alta aplicabilidade em ambientes práticos. 

\section{Resultados do Modelo de Classificação}

A \autoref{tab:metricas322} demonstra os resultados obtidos pelo método de avaliação \textit{COCOEValuator}.


\definecolor{mediumgray}{gray}{0.75} % Cinza mediano
\definecolor{lightgray}{gray}{0.9}   % Cinza claro



\begin{table}[!htbp]
	\footnotesize
	\centering
	\caption{Métricas Globais de Precisão Média (\textit{Average Precision} - AP) do Modelo de Classificação.}
	
	% 4 colunas: l X X X
	\begin{tabularx}{\textwidth}{l X X X}
		\hline
		\rowcolor{mediumgray}
		\textbf{Métrica} & \textbf{Descrição} & \textbf{Detecção (bbox AP)} & \textbf{Segmentação (segm AP)} \\
		\hline
		
		\rowcolor{white}
		$\text{AP}$ & AP Principal ($\text{IoU}=0.50$ a $0.95$) & $\mathbf{88.128\%}$ & $\mathbf{94.243\%}$ \\
		
		\rowcolor{lightgray}
		$\text{AP}^{50}$ & AP com $\text{IoU} \ge 0.50$ (Detecção "Fácil") & $100.000\%$ & $100.000\%$ \\
		
		\rowcolor{white}
		$\text{AP}^{75}$ & AP com $\text{IoU} \ge 0.75$ (Detecção "Rigorosa") & $100.000\%$ & $100.000\%$ \\
		
		\rowcolor{lightgray}
		$\text{AP}_M$ & AP para objetos médios ($32^2 < \text{Area} < 96^2$) & $88.762\%$ & $92.673\%$ \\
		
		\rowcolor{white}
		$\text{AP}_L$ & AP para objetos grandes ($\text{Area} > 96^2$) & $88.960\%$ & $95.384\%$ \\
		\hline
	\end{tabularx}
	\caption*{Fonte: Elaboração Própria (Baseado nos resultados do COCOEvaluator).}
	\label{tab:metricas322}
\end{table}

Dessa forma, é possível destrinchar os elementos dessa tabela:

\begin{itemize}
	\item \textbf{AP (IoU=0.50 a 0.95, média)}: O modelo atinge uma alta precisão no geral, mas possui mais facilidade com a segmentação;
	
	\item \textbf{{$\text{AP}^{50}$} (IoU$\geq$0.50)}: Há uma localização perfeita, o que demonstra que houve uma memorização do conjunto de validação;
	
	\item \textbf{{$\text{AP}^{75}$} (IoU$\geq$0.75)}: Apesar do limiar rigoroso, a localização perfeita permanece, algo que corrobora com a memorização do conjunto de validação;
	
	\item \textbf{{$\text{AP}_{M}$} (Objetos Médios)}: Grande perfomance, especialmente na segmentação;
	
	\item \textbf{{$\text{AP}_{L}$} (Objetos Grandes)}: O modelo lida melhor com objetos de interesse grandes;
\end{itemize}

O modelo de classificação apresentou resultados perfeitos, ou seja, alcançou 100\% em {$\text{AP}^{50}$} e {$\text{AP}^{75}$} no conjunto de validação. No entanto, houve uma queda considerável para 88.128\% (Detecção) e 94.243\% (Segmentação) na métrica de AP principal e mais exigente. De acordo com \citeonline{ying2019overview}, esse contraste entre as métricas é um claro indicativo de que o modelo memorizou os objetos de interesse do conjuto de validação ao invés de generalizar seu aprendizado. 

Além do mais, as métricas com porcentagens perfeitas indica que o principal problema foi o aprendizado de ruído \cite{ying2019overview}. Esse problema ocorre quando o \textit{dataset} de treinamento não possui um tamanho adequado para as necessidades do modelo. Essa escassez aumenta consideravelmente as chances de acontecer uma memorização. Embora uma estratégia de aumento de dados tenha sido aplicada, ela não foi suficiente. 






	
	\chapter{Conclusão}
	\label{chapter:conclusao}
	A construção de um modelo adequado de aprendizado de máquina para identificação de oócitos bovinos e a análise dos problemas enfrentados para a construção do modelo de classificação são as principais contribuições desse trabalho. Isso representa um grande avanço para automatizar a etapa de classificação da PIV, algo de grande importância para a economia brasileira. 

Para um bom desenvolvimento deste trabalho, técnicas de aprendizado de máquina e visão computacional foram estudadas e aplicadas com rigor técnico para que o máximo pudesse ser extraído dos recursos disponíveis para o trabalho. Além do mais, houve a necessidade de obter uma compreensão de técnicas de processamento digital de imagens, que compõem um ramo da visão computacional, para que os treinamentos pudessem ocorrer adequadamente. 

A partir desses estudos e aplicações, conclui-se que há um grande potencial para a capacidade de generalização de métodos aplicada a reprodução animal assistida tendo em vista os ótimos resultados obtidos para o modelo de identificação. Adicionalmente, redes neurais convolucionais mostraram-se de extrema importância para o bom aprendizado dos treinamentos. Mais especificamente, a Mask R-CNN demonstrou-se com uma técnica bastante pertinente para o trabalho por conta da sua característica de segmentação de instância que permite delimitar os contornos dos oócitos ao nível de \textit{pixel}.

Dito isso, a principal conclusão acerca dos resultados o modelo de classificação é que ele possui um problema de \textit{overfitting}, ou seja, o modelo não conseguiu generalizar seu aprendizado. Ele apenas memorizou as imagens que foram usadas para validá-lo.

Portanto, nota-se como nenhuma das estratégias foi suficiente para resolver o \textit{overfitting}. Conclui-se que, para resolver esse problema, é necessário que haja um \textit{dataset} consideravelmente maior e que contenha imagens de alta qualidade dos oócitos isolados. A empresa parceira, BioInova, não conseguiu proporcionar essas imagens em tempo hábil e não havia como realizar essa coleta por conta própria tendo em vista que não há estrutura para isso na região.


	
	
	
	
	% ---
	% Finaliza a parte no bookmark do PDF, para que se inicie o bookmark na raiz
	% ---
	\bookmarksetup{startatroot}% 
	% ---
	
	% ----------------------------------------------------------
	% ELEMENTOS PÓS-TEXTUAIS
	% ----------------------------------------------------------
	\postextual
	
	% ----------------------------------------------------------
	% Referências bibliográficas Título da Pesquisa em Inglês
	% ----------------------------------------------------------
	\bibliography{references}
	
	% ---------------------------------------------------------------------
	% GLOSSÁRIO
	% ---------------------------------------------------------------------
	
	% Arquivo que contém as definições que vão aparebovinoscer no glossário
	%\input{tex/glossario}
	% Comando para incluir todas as definições do arquivo glossario.tex
	%\glsaddall
	% Impressão do glossário
	%\printglossaries
	
	% ----------------------------------------------------------
	% Apêndices
	% ----------------------------------------------------------
	
	% ---
	% Inicia os apêndices
	% ---
%	\begin{apendicesenv}
		
		% Apêndice A
		%\chapter{Título do Apêndice A}
		%\label{chapter:apendicea}
		
		% Apêndice B
		%\chapter{Título do Apêndice B}
		%\label{chapter:apendiceb}
		
	%\end{apendicesenv}
	% --- 
	
	
	% ----------------------------------------------------------
	% Anexos
	% ----------------------------------------------------------
	
	% ---
	% Inicia os anexos
	% ---\\
	%\begin{anexosenv}
		
		%\chapter{Título do Anexo A} 
		%\label{chapter:anexoa}
		
		% CONFIGURAÇÕES DE PDF
		%\section{CONFIGURAÇÕES DE PDF}
		%
		%inserindo uma página em branco depois da página 1
		%\includepdf[pages={1,{},2-10}]{arquivo03.pdf}
		
		%inserindo múltiplas páginas numa única página
		%\includepdf[pages={286-291},nup=2x3]{arquivo05.pdf}
		
		%\chapter{Título do Anexo B} 
		%\label{chapter:anexob}
		
	%\end{anexosenv}
	% ---
	
	
\end{document}